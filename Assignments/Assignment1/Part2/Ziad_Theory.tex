\documentclass{article}
\usepackage[utf8]{inputenc}
\usepackage[margin = 1.1in, headheight = 0.9in, footskip = 0.75 in]{geometry}
\usepackage{fancyhdr}
\usepackage{lastpage}
\usepackage{amsmath}
\usepackage{amssymb}
\usepackage{xparse}
\usepackage{graphicx}
\usepackage{enumerate}
\usepackage{mathtools}
\usepackage{amsthm}
\usepackage{tikz-cd}
\usepackage{tikz}
\usepackage{pgfplots}
\usepackage[shortlabels]{enumitem}
\newcommand{\RNum}[1]{\uppercase\expandafter{\romannumeral #1\relax}}
\newcommand\ddfrac[2]{\frac{\displaystyle #1}{\displaystyle #2}}
\usepackage{xpatch}
\setlength{\parindent}{3em}
\setlength{\parskip}{0.5em}
%===================================================================================================
\pagestyle{fancyplain}
\renewcommand{\headrulewidth}{0pt}
\renewcommand{\footrulewidth}{0pt}
\fancyhf{}\par 
\lhead{\hspace{0cm}\\\hspace{0cm}\\Zmh6339\\AI \& Machine Learning, CS-UH 3260\\Professor Keith Ross}
\rhead{Ziad Hassan\\January 31, 2024}
\cfoot{\thepage/\pageref{LastPage}}
%===================================================================================================
\title{Assignment 1}
\date{}
\author{}
%===================================================================================================
\begin{document}
\maketitle
%==================================================================================================
\section*{Part 1: Continuous Bandit Algorithm}
%==================================================================================================
\section*{Part 2: Theory}
\begin{enumerate}[a)]
    \item \begin{proof}
        \renewcommand{\qedsymbol}{$\blacksquare$}
        \hfill\\\\
        For the algorithm to consider taking the \textit{one} greedy action, two scenarios must be taken into consideration.
        \\\\
        The first scenario is that the algorithm decides to take the greedy action, which occurs with probability $1\cdot(1 - \epsilon)$, where $\epsilon$ is the probability of taking a random action, and $1$ is the probability of taking the one greedy action.
        \\\\
        The second scenario is that the algorithm decides to take a random action, which occurs with probability $\epsilon$.\\
        Additionaly, the algorithm chooses a random action with an equal probability for each action; so, the probability of choosing the greedy action is $\frac{\epsilon}{k}$, where $k$ is the number of actions.
        \\\\
        Given that the greedy action can be chosen during exploration \textit{or} exploitation, the above probabilities must be added together.\\
        Therefore, the probability of the algorithm taking the greedy action is $(1 - \epsilon) + \frac{\epsilon}{k}$.\par 
    \end{proof}
    %==============================================
    \item  
    \begin{enumerate}[i)]
        \item \begin{proof}
            \renewcommand{\qedsymbol}{$\blacksquare$}
            \hfill\\\\
            To determine the probability that the greedy action was chosen for the first time at time $T$, we need to consider that it was not chosen at any time before $T$, and that it was chosen at time $T$.\\
            Thus, the following equation should be quantified:
            \begin{equation*}
                P(\text{greedy at } T) = P(\text{not greedy before } T) \cdot P(\text{greedy at } T)
            \end{equation*}
            Therefore, the probability that the greedy action was chosen for the first time at time $T$ is:
            \begin{equation*}
                \begin{aligned}
                    P(\text{greedy at } T) &= P(\text{not greedy before } T) \cdot P(\text{greedy at } T)\\
                    &= \left(1 - 1 + \epsilon - \frac{\epsilon}{k}\right)^{T - 1} \cdot \left(1 - \epsilon + \frac{\epsilon}{k}\right)\\
                    &= \left(\epsilon - \frac{\epsilon}{k}\right)^{T - 1} \cdot \left(1 - \epsilon + \frac{\epsilon}{k}\right)\\
                \end{aligned}
            \end{equation*}\par
        \end{proof}

    \item \begin{proof}
        \renewcommand{\qedsymbol}{$\blacksquare$}
        \hfill\\\\
        To get the expected number of steps, $\mathbb{E}(T)$, until the the greedy action is chosen for the first time is a sum over all possible time steps, each weighted by its probability of being the first time the greedy action is chosen.\\
        Thus, the following equation should be quantified:
        \begin{equation*}
            \mathbb{E}(T) = \sum_{t = 1}^{\infty} t \cdot P(\text{greedy at } t)
        \end{equation*}
        Following, the equation is 
        \begin{equation*}
            \begin{aligned}
                \mathbb{E}(T) &= \sum_{t = 1}^{\infty} t \cdot P(\text{greedy at } t)\\
                &= \sum_{t = 1}^{\infty} t \cdot \left(\epsilon - \frac{\epsilon}{k}\right)^{t - 1} \cdot \left(1 - \epsilon + \frac{\epsilon}{k}\right)\\
            \end{aligned}
        \end{equation*}
        It can be observed that the above equation is a geometric series, which can be simplified to the following:
        \begin{equation*}
            \begin{aligned}
                \mathbb{E}(T) &= \sum_{t = 1}^{\infty} t \cdot \left(\epsilon - \frac{\epsilon}{k}\right)^{t - 1} \cdot \left(1 - \epsilon + \frac{\epsilon}{k}\right)\\
                &= \ddfrac{1}{\left(\epsilon - \frac{\epsilon}{k}\right)^{t - 1} \cdot \left(1 - \epsilon + \frac{\epsilon}{k}\right)}
            \end{aligned}
        \end{equation*}
        The above simplifciation is valid due to the definiton of the expected value of a geometric series.\par 
    \end{proof}
    \end{enumerate}
\end{enumerate}
%==================================================================================================
\end{document}